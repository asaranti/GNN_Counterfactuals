\documentclass[12p3, letterpaper, twoside]{article}
\usepackage[utf8]{inputenc}
\usepackage{hyperref}
\usepackage{graphicx}
\usepackage{circuitikz}
\usepackage{tikz}
\usetikzlibrary{positioning}
\tikzset{block/.style = {draw, rectangle},
Connect/.style={-latex,thick}}

%%%%%%%%%%%%%%%%%%%%%%%%%%%%%%%%%%%%%%%%%%%%%%%%%%%%%%%%%%%%%%%%%%%%

\title{GNN Counterfactual Specifications}
\date{05. October 2021}

%%%%%%%%%%%%%%%%%%%%%%%%%%%%%%%%%%%%%%%%%%%%%%%%%%%%%%%%%%%%%%%%%%%

\begin{document}

\begin{titlepage}
\maketitle
\end{titlepage}

%%%%%%%%%%%%%%%%%%%%%%%%%%%%%%%%%%%%%%%%%%%%%%%%%%%%%%%%%%%%%%%%%%% 
\section{Introduction}

Current xAI platform constraints influence the GNN counterfactuals corresponingly. Only GNNs for homogeneous graphs (all nodes same number and type of features) and no multi-graphs (no self-edges or multiple edges between nodes). The software package that is used is 
Pytorch-Geometric (PyG):\\ 
\url{https://pytorch-geometric.readthedocs.io/en/latest/}, so the architectures that are already implemented in this package have priority. It needs to be noted that this package (as well as another well-known based on that topic \url{https://www.dgl.ai/}) can be used to create new custom types of GNN architetures. 

Hierarchy levels:

\begin{itemize}
	\item Graph tasks
	\item Typical state-of-the-art architectures
	\item Possible actions
\end{itemize}

The change of number of classes is out of scope. Same goes for the number of hidden units (or so-called hidden channels).

%%%%%%%%%%%%%%%%%%%%%%%%%%%%%%%%%%%%%%%%%%%%%%%%%%%%%%%%%%%%%%%%%%% 
\section{GNN Architectures}

\url{https://pytorch-geometric.readthedocs.io/en/latest/modules/nn.html#models}

GCN is the most prominent, GIN for graph classification

The possible layers are listed here: \url{https://pytorch-geometric.readthedocs.io/en/latest/modules/nn.html#convolutional-layers}

%%%%%%%%%%%%%%%%%%%%%%%%%%%%%%%%%%%%%%%%%%%%%%%%%%%%%%%%%%%%%%%%%%% 
\section{Actions}

\begin{itemize}
	\item \texttt{add\_node}
	\item \texttt{remove\_node}
	\item \texttt{add\_edge}
	\item \texttt{remove\_edge} 
	\item \texttt{add\_feature\_all\_nodes}
	\item \texttt{remove\_feature\_all\_nodes} 
	\item \texttt{add\_feature\_all\_egdes}
	\item \texttt{remove\_feature\_all\_egdes} 
\end{itemize}

%%%%%%%%%%%%%%%%%%%%%%%%%%%%%%%%%%%%%%%%%%%%%%%%%%%%%%%%%%%%%%%%%%%
\section{Quality Management}
\label{Quality Management}

The aforementioned logic needs to be tested after each series of actions is ``submited'', i.e. after the buttons ``Predict'' or ``Retrain'' are pressed. Property-based Testing: \cite{Saranti:2020:PropertyBasedTesting}.

%%%%%%%%%%%%%%%%%%%%%%%%%%%%%%%%%%%%%%%%%%%%%%%%%%%%%%%%%%%%%%%%%%%
\section{Future Work}
\label{Future Work}


%%%%%%%%%%%%%%%%%%%%%%%%%%%%%%%%%%%%%%%%%%%%%%%%%%%%%%%%%%%%%%%%%%%
\bibliographystyle{splncs04}
\bibliography{references}

\end{document}